\documentclass[a4paper, left=1in, right=1in,12pt]{article}
\usepackage[utf8]{inputenc}

\title{Database Management Systems Lab\\Lab 1\\CSE 4308}
\author{\textbf{\Large A Z Hasnain Kabir}\\\Large 200042102\\\Large Software Engineering}
\date{17 August 2022}

\begin{document}

\maketitle

\section*{\large Introduction}
Databases are used virtually everywhere to store, retrieve, manipulate information sometimes on a massive scale. There are many ways to store data on computers. Data can be stored through files such as text files, json files, pdf files etc. But modern technological tools and libraries such as MySql, Oracle XE, NoSql, Pandas etc, have proven to be more efficient and easy to use. In this lab, we are to demonstrate how to work with data using traditional file system.
\section*{\large Database Provided}
Our database contains two files which are 'studentInfo.txt' and 'grades.txt'. In the grades.txt file, we have \textit{Student ID}, \textit{GPA} and the \textit{Semester} in which a student achieved that grade. On the other hand, in the 'studentInfo.txt' file, we have \textit{Student ID, Name, Age, Blood Group} and \textit{Department} of a student. Both of these files contains a semicolon as delimiter(;). 
\section*{Task 1}
\textbf{Task: }We are to calculate the lowest GPA among all the students and print the Student ID of that student.\\
\textbf{Analysis of the problem: }In the grades.txt file, GPA of each student is present. If we can iterate through all of the GPA values in that file, we will surely be able to retrive the information of the student with the lowest GPA.\newline\newline
\textbf{\textit{\large Code:}}\newline
\textbf{Explanation of solution: }A C++ program is created where we define a class called Student with the attributes \textit{Student ID, GPA and Semester}. The grades.txt file is opened by calling \textit{fopen()}. We read each line using the fgets() function where the line is stored in a character array called buffer. We split the line through delimiter(;)  and using the \textit{sscanf()} function, we store the values inside our predefined Student class array. Each value is stored successively in our class attributes.
Then, when storing is complete, we close our file using \textit{fclose()} function and proceed to iterate through the gpa values. After finding the lowest gpa, we proceed to print the Student ID associated with the gpa.
\end{document}
