\documentclass[a4paper, left=1in, right=1in,12pt]{article}
\usepackage[utf8]{inputenc}
\usepackage{minted}
\title{Database Management Systems Lab\\Lab 1\\CSE 4308}
\author{\textbf{\Large A Z Hasnain Kabir}\\\Large 200042102\\\Large Software Engineering}
\date{17 August 2022}

\begin{document}

\maketitle

\section*{\large Introduction}
Databases are used virtually everywhere to store, retrieve, manipulate information sometimes on a massive scale. There are many ways to store data on computers. Data can be stored through files such as text files, json files, pdf files etc. But modern technological tools and libraries such as MySql, Oracle XE, NoSql, Pandas etc, have proven to be more efficient and easy to use. In this lab, we are to demonstrate how to work with data using traditional file system.
\section*{\large Database Provided}
Our database contains two files which are 'studentInfo.txt' and 'grades.txt'. In the grades.txt file, we have \textit{Student ID}, \textit{GPA} and the \textit{Semester} in which a student achieved that grade. On the other hand, in the 'studentInfo.txt' file, we have \textit{Student ID, Name, Age, Blood Group} and \textit{Department} of a student. Both of these files contains a semicolon as delimiter(;). 
\section*{Task 1}
\textbf{Task: }We are to calculate the average GPA among all the students and print the Student ID of that student.\\
\textbf{Analysis of the problem: }In the grades.txt file, GPA of each student is present. If we can iterate through all of the GPA values in that file and store them in a variable, we will surely be able to retrieve the information of the average GPA among the students.\newline\newline
\textbf{\textit{\large Code:}}\newline
\begin{minted}{C++}
#include<iostream>
using namespace std;
class Student{
public:
    int id;
    double gpa=0.00;
    int semester;
};

int main(){

    Student students[100];

    FILE *fp;
    fp = fopen("grades.txt", "r");

    char buffer[160];

    int count = 0;
    int i=0;
    while(fgets(buffer, 159, fp)){

        sscanf(buffer, "%d;%lf;%d", &students[i].id,
        &students[i].gpa, &students[i].semester);
        count++;
        i++;
    }
    //printf("%d", count);

    fclose(fp);

    double total = 0.00f;
    for(int i=0;i<count;i++){
        total+= students[i].gpa;
    }
    double average = total/(count);

    printf("Average gpa is: %lf", average);
}

\end{minted}
\textbf{Explanation of solution: }A C++ program is created where we define a class called Student with the attributes \textit{Student ID, GPA and Semester}. The grades.txt file is opened by calling \textit{fopen()}. We read each line using the fgets() function where the line is stored in a character array called buffer. We split the line through delimiter(;)  and using the \textit{sscanf()} function, we store the values inside our predefined Student class array. Each value is stored successively in our class attributes.
Then, when storing is complete, we close our file using \textit{fclose()} function and proceed to iterate through the gpa values. We store the sum of the gpa in an variable called average which is divided by the total count of students to get the correct value of average.\\
\newline
\textbf{Findings: }Using fscanf() instead of sscanf() skips the first line read by fgets() function. In order to ensure all the values are read, we needed to use sscanf() function.

\section*{Task 2}
\textbf{Task: }We are to take \textit{Student ID, GPA, and Semester} as input. Then we need to add those to our grades.txt file after validating the input.\\
\textbf{Analysis of the problem: }To validate our input we need to check whether the GPA is within 2.0 - 4.0, Semester is within 1 - 8 and StudentID are only integer.\newline\newline
\textbf{\textit{\large Code:}}\newline
\begin{minted}{C++}
#include<iostream>
using namespace std;

class Student{
public:
    int id;
    double gpa=0.00;
    int semester;
};

int main(){

    Student students[100];

    FILE *fp;
    fp = fopen("grades.txt", "r");

    char buffer[124];

    int count = 0;
    int i=0;
    while(fgets(buffer, 123, fp)){

        fscanf(fp, "%d;%lf;%d", &students[i].id, &students[i].gpa,
        &students[i].semester);
        count++;
        i++;
    }
    //printf("%d", count);

    int id,semester;
    float gpa;

    cout<<"Enter your student ID: "<<endl;
    cin>>id;

    cout<<"Enter your gpa: "<<endl;

    while(cin>>gpa){
        if(gpa<=4.0 && gpa >=2.00){
            break;
        }else{
            cout<<"Invalid gpa"<<endl;
            cout<<"Enter your gpa: "<<endl;
        }
    }

    cout<<"Enter your semester: "<<endl;

    while(cin>>semester){
        if(semester<=8){
            break;
        }else{
            cout<<"Invalid semester"<<endl;
            cout<<"Enter your semester: "<<endl;
        }
    }

    students[count-1].id = id;
    students[count-1].gpa = gpa;
    students[count-1].semester = semester;


    fclose(fp);

    fp = fopen("grades.txt", "a");

    fprintf(fp, "\n%d;%.2f;%d", students[count-1].id,
    students[count-1].gpa, students[count-1].semester);
    fclose(fp);
}


\end{minted}
\textbf{Explanation of solution: }A C++ program is created where we define a class called Student with the attributes \textit{Student ID, GPA and Semester}. The grades.txt file is opened by calling \textit{fopen()}. We read each line using the fgets() function where the line is stored in a character array called buffer. We split the line through delimiter(;)  and using the \textit{sscanf()} function, we store the values inside our predefined Student class array. Each value is stored successively in our class attributes.
Then, when storing is complete, we close our file using \textit{fclose()} function and proceed to validate our input variables using while loops. Then we store the values in our Student class array at the very end of the array. Then we open grades.txt file in append mode and print our values to that file.

\section*{Task 3}
\textbf{Task: }We are to take \textit{Student ID} as input. Then we need to show his/her name with the lowest GPA he/she got in a semester.\\
\textbf{Analysis of the problem: }We can iterate through Student array to find the student with the given \textit{StudentID} and then iterate through all the GPA to find the lowest GPA in all semester.\newline\newline
\textbf{\textit{\large Code:}}\newline
\begin{minted}{C++}
#include<iostream>
#include<new>
using namespace std;

class Studentgrade{
public:
    int id;
    double gpa=0.00;
    int semester;

};

class Student{
public:
    int id;
    char* name;
    int age;
    char* blood_group;
    char* dept;
};

int main(){

    Studentgrade studentgrade[100];

    FILE *fp;
    fp = fopen("grades.txt", "r");

    char buffer[161];

    int count = 0;
    int i=0;
    while(fgets(buffer, 160, fp)){

        sscanf(buffer,"%d;%lf;%d", &studentgrade[i].id,
        &studentgrade[i].gpa, &studentgrade[i].semester);
        count++;
        i++;
    }
    //printf("%d", count);

    fclose(fp);

    fp = fopen("studentInfo.txt", "r");

    int cnt = 0;

    Student students[100];

    int j=0;
    while(fgets(buffer, 160, fp)){
        sscanf(buffer, "%d;%s;%d;%s;%s", &students[j].id,
        students[j].name, &students[j].age, students[j].blood_group,
        students[j].dept);
        j++;
    }

    fclose(fp);

    int stdId=0;

    cout<<"Enter your student ID: "<<endl;
    cin>>stdId;


    for(int x=0;x<j;x++){
        printf("%s\n", students[x].name);
    }
}

\end{minted}
\textbf{Explanation of solution: }A C++ program is created where we define a class called Student with the attributes \textit{Student ID, GPA and Semester}. The grades.txt file is opened by calling \textit{fopen()}. We read each line using the fgets() function where the line is stored in a character array called buffer. We split the line through delimiter(;)  and using the \textit{sscanf()} function, we store the values inside our predefined Student class array. Each value is stored successively in our class attributes.
Then, when storing is complete, we close our file using \textit{fclose()} function. Then take StudentID as input then iterate through all the values to find the lowest GPA of all the semesters. \\
\newline
\textbf{Findings within code: }The program, when running throws a bad alloc exception for bad memory allocation. This prevents the code from working and therefore, we could not find the required value through our solution.
\newline
\end{document}
