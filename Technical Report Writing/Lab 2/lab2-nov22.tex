\documentclass[a4paper, 12pt]{article}
\usepackage[margin=1in]{geometry}
\usepackage{hyperref}

\usepackage{amsmath, amssymb}

\title{CSE 4714 (SWE) 2024 Section B Lab 2}
\author{Mohammad Ishrak Abedin}
\date{November 2024}

\begin{document}

\maketitle

\section{Modes}
First of all, there are two modes. 

\subsection{Inline math}
We can write a mathematical expression as, $             (a + b)^2 =\qquad a^2 + 2ab + b^2 $. We can also write inline maths as \( (a + b)^2 = a^2 + 2ab + b^2 \). Or,
\begin{math}
    (a + b)^2 = a^2 + 2ab + b^2
\end{math}.

\subsection{Display Math} \label{ssec:displaymath}
We can create display maths as, \[ (a + b)^2 = a^2 + 2ab + b^2 \] We can also write display maths with,
\begin{displaymath}
    (a + b)^2 = a^2 + 2ab + b^2
\end{displaymath}
Or, with,
\begin{equation}
    a + b = 2
\end{equation}
\begin{equation}
    a + b = 2
\end{equation}
\begin{equation} \label{eqn:aplusbwholesquare}
    (a + b)^2 = a^2 + 2ab + b^2
\end{equation}

In \autoref{ssec:displaymath}, in \autoref{eqn:aplusbwholesquare}, we can find the formula for a plus b whole square.

\section{Subscript and Superscript}
\begin{equation}
    X^{2000}_{(i,j)} = 10
\end{equation}

\section{Math Operators}
\begin{equation}
    1 + 2 = 5 - 2 = 6 / 2 = 1.5 \times 2
\end{equation}
\begin{equation}
    sin^2\theta + cos^2\theta = 1
\end{equation}
\begin{equation}
    \sin^2\theta + \cos^2\theta = 1 = \log{10} \text{, this is a simple equation}
\end{equation}

We can write, an inline equation, $\frac{a}{b} = \frac{5}{10}$
\begin{equation}
    \frac{a}{b} = \frac{5}{10}
\end{equation}


$$\frac{a}{b} = \frac{5}{10}$$

We can write, an inline equation, $\displaystyle \frac{a}{b} = \frac{5}{10}$. We can write, an inline equation,  $\displaystyle\frac{a}{b} = \frac{5}{10}$.We can write, an inline equation, $\frac{a}{b} = \frac{5}{10}$.We can write, an inline equation, $\frac{a}{b} = \frac{5}{10}$. We can write, an inline equation, $\frac{a}{b} = \frac{5}{10}$. We can write, an inline equation, $\frac{a}{b} = \frac{5}{10}$. We can write, an inline equation, $\frac{a}{b} = \frac{5}{10}$.We can write, an inline equation, $\frac{a}{b} = \frac{5}{10}$.We can write, an inline equation, $\frac{a}{b} = \frac{5}{10}$. We can write, an inline equation, $\frac{a}{b} = \frac{5}{10}$.


\section{Brackets}
\begin{equation}
    \left.
    \sqrt{\frac{ \frac{a + 5 \times 3}{ \frac{2}{3} } }{ \frac{2}{3} }} + 5 
    \right\}
    \times 3 = \frac{1}{ \sqrt{2} }
\end{equation}

\section{Greek Alphabets}
\begin{equation}
    \alpha A \beta B \gamma\Gamma\delta\Delta
\end{equation}

\section{Calculus}
\begin{equation}
    \lim_{x \rightarrow \infty} \frac{1}{x} = 0
\end{equation}

\begin{equation}
    \int_{10}^{200} x dx = 500
\end{equation}

\begin{equation}
    \int\limits_{10}^{200} x dx = 500
\end{equation}

\begin{equation}
    \sum_{i = 10}^{200} x_i = 500 = \prod_{i = 10}^{200} x_i
\end{equation}

\section{Calligraphy}
\begin{equation}
    \mathcal{ABCDEF}, \mathfrak{ABCDEF}, \mathbb{ABCDEF}
\end{equation}

\section{Comparison Operators}
\begin{equation}
    1 = 1 < 5 > 2 \neq 4 \leq 5 \geq 3 \ngtr 5
\end{equation}

\section{Sets and Vectors}
\begin{equation}
    A \cup B \cap C \in \mathbb{R}
\end{equation}
\begin{equation}
    \Vec{A} = \Vec{B}
\end{equation}
\begin{equation}
    \hat{i} \times \hat{j} = \hat{k}
\end{equation}
\begin{equation}
    \Vec{A} \cdot \Vec{B} = 5.32 
\end{equation}

\section{AMS Math Environments}
\begin{equation*}
    a + b = 3
\end{equation*}

\begin{multline}
    1 + 2 + 3 + 4 + 1 + 2 + 3 + 4 + 1 + 2 + 3 + 4 + \\
    1 + 2 + 3 + 4 + 1 + 2 + 3 + 4 + 1 + 2 + 3 + 4 + \\
    1 + 2 + 3 + 4 + 1 + 2 + 3 + 4 + 1 + 2 + 3 + 4 + \\
    1 + 2 + 3 + 4 + 1 + 2 + 3 + 4 + 1 + 2 + 3 + 4 + ... = \infty
\end{multline}

\begin{equation}
    \begin{split}
        \frac{a}{b} &= \frac{5}{10} \\
                    &= \frac{1}{2}  \\
                    &= 0.5
    \end{split}
\end{equation}

\begin{align}
    a + b + c &= 5 \\
    a - 2b + 4c &= 15 \\
    -5a - 2b + 7c &= -5
\end{align}

\begin{gather}
    a + b + c = 5 \\
    a = 15 \\
    - 2b + 7c = -5
\end{gather}

\section{Matrices}
\begin{equation*}
    I = \begin{bmatrix}
        1 & 0 & 0 \\
        0 & 1 & 0 \\
        0 & 0 & 1 \\
    \end{bmatrix}
\end{equation*}
\end{document}
