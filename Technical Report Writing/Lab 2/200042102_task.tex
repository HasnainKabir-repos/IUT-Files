\documentclass[a4paper,12pt]{article}
\usepackage[margin=1in]{geometry}
\usepackage{hyperref}
\usepackage{amsmath, amssymb}

\title{Katugampola Fractional Derivative}
\author{Student ID: 200042102}
\date{February 20, 2025}

\begin{document}
	
	\maketitle
	
	\section{Derivation of Katugampola Fractional Derivative}
	The Katugampola Fractional Derivative is a generalization of classical derivative as defined in \autoref{eq:1}
	\begin{equation} \label{eq:1}
		D^\alpha_a f(x) = \frac{1}{\Gamma(m-\alpha)} \frac{d^m}{dx^m} \int_{0}^{x} (x-t)^{m-\alpha-1} f(t) \, dt,
	\end{equation}
	where $m=\lceil a \rceil$ and $\alpha \in (0,1)$.
	
	\begin{equation} \label{eq:2}
		I^\alpha_{0+} f(x) = \frac{1}{\Gamma(\alpha)} \int_{0}^{x} (x-t)^{\alpha-1} f(t) \, dt,
	\end{equation}
	with $0 < \alpha < 1$.
	
	\begin{equation} \label{eq:3}
		\begin{split}
			D^\alpha_a f(x) &= \frac{1}{\Gamma(m-\alpha)} \frac{d^m}{dx^m} \textit{I}^{\alpha}_{0+} f(x) \\
			&= \frac{1}{\Gamma(m-\alpha)} \frac{d^m}{dx^m} \left(\frac{1}{\Gamma(\alpha)} \int_{0}^{x} (x-t)^{\alpha-1} f(t) \, dt \right).
		\end{split}
	\end{equation}
	
	\section{Conclusion}
	Replicating references to (\ref{eq:1}),  (\ref{eq:2}),  (\ref{eq:3})
	
\end{document}
