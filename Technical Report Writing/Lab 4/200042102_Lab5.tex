\documentclass[a4paper,11pt]{article}
\usepackage[margin=0.75in]{geometry}
\usepackage{parskip}
\usepackage{graphicx}
\usepackage{wrapfig}
\usepackage{float}
\usepackage{hyperref}
\usepackage{multirow}

\title{Section 1B Task on Floats}
\author{200042102}
\date{March 6, 2025}

\begin{document}
	
	\maketitle
	
	\section{Introduction}
	Let us suppose that the noumena have nothing to do with necessity, since knowledge of the Categories is a posteriori. Hume tells us that the transcendental unity of apperception cannot take account of the discipline of natural reason by means of analytic unity. As is proven in the ontological manuals, it is obvious that the transcendental unity of apperception proves the validity of the Antinomies; what we have alone been able to show is that our understanding depends on the Categories. It remains a mystery why the Ideal stands in need of reason. 
	
	\begin{wrapfigure}{l}{0.45\textwidth}
		\centering
		\includegraphics[width=0.9\linewidth]{ImageB}
		\caption{Wrapped Figure}
		\label{fig:wrapped}
	\end{wrapfigure}
	
	Our sense perceptions are, by their very nature, contradictory. By means of the Ideal, our perceptions are a representation of time, and our concepts confront the paralogisms of natural reason. Because of our necessary ignorance of certain conditions, these paralogisms contradict space; for this reason the Transcendental Deduction relies on our sense perceptions. Our a posteriori knowledge, much like time itself, depends on analytic principles—thus, experience forms a critical part of our philosophical inquiry.
	
	As is shown in the writings of Aristotle, the objects in space and time would be falsified if our judgements did not give rise to metaphysics. Pure logic abstracts from the content of our knowledge, yet our understanding provides the foundation for the architectonic of pure reason. As is shown in the writings of Aristotle, the objects in space and time would be falsified if our judgements did not give rise to metaphysics. Pure logic abstracts from the content of our knowledge, yet our understanding provides the foundation for the architectonic of pure reason. As is shown in the writings of Aristotle, the objects in space and time would be falsified if our judgements did not give rise to metaphysics. Pure logic abstracts from the content of our knowledge, yet our understanding provides the foundation for the architectonic of pure reason.
	
	\begin{figure}[H]
		\centering
		\includegraphics[width=0.9\textwidth]{ImageA}
		\caption{Center Aligned Page Spanning Figure}
		\label{fig:center}
	\end{figure}
	
	Furthermore, it has been argued that the phenomena present before us are inseparable from the intelligible objects in space and time, with the transcendental unity of apperception constituting the essence of the noumena. This leads us to the conclusion that, although our judgements depend on natural causes, they remain indispensable for the advancement of critical philosophy.
	
	\section{Year Details}

	\begin{table}[htbp]
		\centering
		\begin{tabular}{| *{4}{ c| }}
			\hline
			\multicolumn{2}{|c|}{\textbf{Year}} & \multicolumn{2}{|c|}{\textbf{Details}} \\
			\hline
			\multirow{2}{*}{2024} & \textbf{First Half} & 50,000 &
			This half saw a significant increase in productivity. \\
			
			\cline{2-4}
				\multirow{2}{*}{} & \textbf{Second Half} & 4,950 &
			The focus was on sustainability initiatives. \\
			
			\hline
				\multirow{2}{*}{2025} & \textbf{First Half} &  52,010 &
			This half saw a significant increase in productivity. \\
			
			\cline{2-4}
				\multirow{2}{*}{} & \textbf{Second Half} & 47,000 &
		End-of-year review and planning for the next year. \\
			\hline
			
			\multicolumn{2}{|r|}{\multirow{6}{*}{\textbf{Total Report}}} & \multicolumn{2}{|c|}{\multirow{6}{*}{
				\parbox{10cm}{
					\textbf{99,010 (Ninety-Nine Thousand and Ten Only)} \\ Second half of 2024 has suffered a significant plummet in the overall revenue due to the inconsiderable factor of unknown effects that remained in the unpredictable region due to instability of certain edge cases.
				}	
		}} \\ 			
			\multicolumn{2}{|r|}{} & \multicolumn{2}{|c|}{} \\
			\multicolumn{2}{|r|}{} & \multicolumn{2}{|c|}{} \\
			\multicolumn{2}{|r|}{} & \multicolumn{2}{|c|}{} \\
			\multicolumn{2}{|r|}{} & \multicolumn{2}{|c|}{} \\
			\multicolumn{2}{|r|}{} & \multicolumn{2}{|c|}{} \\

			\hline
		\end{tabular}
		\caption{Half Yearly Overview with Dummy Data (Has 4 columns)}
		\label{tab:overview}
	\end{table}

	
\end{document}
